\subsection{Sudo-licho města}


\df Nechť $|X|=n$ a $A_1, \dots, A_m \subseteq X$ $A_i \ne A_j$  jsou neprázdné podmnožiny.  
Úloha A-B město se ptá, jak velké může být $m$, pokud $|A_i| \sim B$ a $|A_i\cap 
A_j|\sim A$ (tedy pro sudo-licho město máme omezení na liché velikosti a sudé 
průniky).

\vt Pro úlohu sudo-licho město platí $m \leq n$.

\dk Počítejme nad $GF(2)$. Matice $A$ nechť je charakteristická matice dimenze 
$n \times m$. Podívejme se na součin $AA^T$, tedy na matici skalárních součinů:
\begin{align}
	AA^T = \left(\begin{matrix}A_1\\ A_2 \\ \vdots \\ A_m \end{matrix}\right) 
	\cdot \left(\begin{matrix}A_1, A_2, \dots, A_m\end{matrix}\right) =
	\left(\begin{matrix}
	1 & &\bigzero & \\
	& \ddots && \\
	&\bigzero& \ddots & \\
	&&  &1
	\end{matrix}\right)
\end{align}
Tedy víme, že $\rank(AA^T) = m$ a $\rank(A) \leq n$. Z vlastností ranku již 
snadno získáme nerovnost $m=\rank(AA^T) \leq \rank(A) \leq n$. \todo{důkaz 
rankové nerovnosti obrázkem pomocí zobrazení} \qed

\vt Nechť $|X|=n$ a $A_1, \dots, A_m \subseteq X$ že platí $|A_i \cap A_j| = 1 $ 
a $A_i \neq A_j$. Potom $m \leq n$.
\dk Podobně jako v předchozím příkladě vezměme matici charakteristických vektorů 
$A$ a podívejme se na součit $AA^T$, tentokrát již nad $Q$:
\begin{align}
	AA^T = \left(\begin{matrix}
	|A_1| & &\bigone & \\
	& \ddots && \\
	&\bigone& \ddots & \\
	&&  &|A_m|
	\end{matrix}\right)
\end{align}
Dále označme $a_i := |A_i|$. Můžeme předpokládat, že $a_1 \leq a_2 \leq \dots 
\leq a_m$. Zřejmě také $a_2 > 1$ (jinak $A_1 = A_2$). Nyní bychom chtěli 
dokázat, že je matice regulární -- proto se podíváme na determinant této matice:
\begin{align}
	|AA^T| = \left|\left(\begin{matrix}
	a_1 & &\bigone & \\
	& \ddots && \\
	&\bigone& \ddots & \\
	&&  &a_m
	\end{matrix}\right)\right|
	%= \left|\begin{matrix}\text{{\fontsize{120}{120}\selectfont 1}}\end{matrix}
	= \left|\begin{matrix}\text{\bf\Huge 1}\end{matrix}
	+\left(\begin{matrix}
	a_1-1 & &\bigzero & \\
	& \ddots && \\
	&\bigzero& \ddots & \\
	&&  &a_m-1
	\end{matrix}\right)\right|
\end{align}
Zatímco matice jedniček je singulární \todo{Pochopit proč se to dá spočítat, ale 
determinant vyjde kladně}.  \qed

