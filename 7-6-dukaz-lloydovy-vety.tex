\subsection{Důkaz Lloydovy věty}


Zde začnou věci dávat větší smysl. Nejdříve dokážeme pomocí výše zmíněných lemat pomocné tvrzení, který dá podobný polynom, následně si s ním pohrajeme a získáme polynom Lloydův, tak jak byl zadefinován na začátku.

\vt (Lloydův prototyp) Pokud existuje $t$-perfektní kód v $G$, potom $x_t(\lambda)\backslash x_d(\lambda)$.

\dk Nejprve si všimněme, že $\widehat{S_t} = \widehat{X_t(A)} = \widehat{\sum_i^t A_i} = \sum_i^t B_i = X_t(B)$.
Dále se podívejme na spektra $B$ a $\widehat{S_t}$:
\begin{align}
	&\Sp(B) = \{ k, \lambda_1, \ldots, \lambda_d \} \\
	&\Sp(\widehat{S_t}) = \{ x_t(k), x_t(\lambda_1), \ldots, x_t(\lambda_d) \}
\end{align}

\todo{proc a zbytek...}


