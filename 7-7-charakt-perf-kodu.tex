\subsection{Charakterizace perfektních kódů}


\vt Nechť $q=p^r$, a $p$ je prvočíslo. Pak existují právě následující netriviální perfektní kódy (tedy s $|C| \geq 2$ a pokud $|C| = 2$, tak to není kód $q=2$ a $n=2t+1$):
\begin{description}
	\item $1$-perfektní kód $n={q^k-1 \over q-1}$ pro libovolné $k$ (Hammingův)
	\item $2$-perfektní kód $q=3$ a $n=11$ (Golayův)
	\item $3$-perfektní kód $q=2$ a $n=23$ (Golayův)
\end{description}
Dál $q$ složené neexistují perfektní kódy pro $t \geq 3$ a pro $t = 1,2$ se to neví.

\dk Důkaz je technicky náročný. Základ je v Lloydově větě, která dává relativně silný nástroj jak perfektní kód poznat. Společně se hrubým odhadem na velikost kódu ukázaným na začátku sekce, lze pomocí hrubé síly a netriviální teorie čísel získat výsledek. Ten však není v naší moci.


