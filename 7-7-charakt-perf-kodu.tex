\subsection{Charakterizace perfektních kódů}

\vt Nechť $q=p^r$, a $p$ je prvočíslo. Pak existují právě následující netriviální perfektní kódy (tedy s $|C| \geq 2$ a pokud $|C| = 2$, tak to není kód $q=2$ a $n=2t+1$):
\begin{description}
	\item $1$-perfektní kód $n={q^k-1 \over q-1}$ pro libovolné $k$ a $q$ (Hammingův)
	\item $2$-perfektní kód $q=3$ a $n=11$ (Golayův)
	\item $3$-perfektní kód $q=2$ a $n=23$ (Golayův)
\end{description}
Dál $q$ složené neexistují perfektní kódy pro $t \geq 3$ a pro $t = 1,2$ se to neví.

Důkaz je technicky náročný a budeme se jím zabývat po zbytek sekce. Základem je Lloydova věta a Sphere packing ukázaným na začátku sekce. Nejprve se pro malé hodnoty parametrů ukáže, zda pro dané hodnoty kódy existují či nikoli. Pak se pro obecný případ udělá horní odhad parametrů pomocí Lloydovy věty. A pro konečný počet případů, pro které by kódy mohly existovat, bylo dokázáno počítačem, že jiné perfektní kódy než Hammingovy a Golayovy neexistují.

\vt Pro $q=3$ existuje jenom $2$-perfektní kód.

\vt Pro $q=2$ neexistuje $2$-perfektní kód.

\dk Ze Sphere packingu dostaneme:
\begin{align}
	1 + n + {n \choose 2} &= q^\alpha  \\
	2 + 2n + n(n-1) &= q^{\alpha + 1} \label{eq:SpherePacking}\\
	7 + (2n + 1)^2 &=q^{\alpha + 3} \label{eq:SpherePacking2}
\end{align}

A dále pak z Lloydovy věty dostaneme:
\begin{align*}
L_2(x) &= {n - x \choose 2} - (x - 1)(n -x) + {x - 1 \choose 2} \\
2L_2(x) &= n^2 + n + 2 + 4x^2 - 2(n + 1)2x
\end{align*}

Provedeme substituci $y = 2x$ a za $n^2 + n + 2$ dosadíme $q^{\alpha + 1}$ (z rovnice~\ref{eq:SpherePacking}):
\begin{align*}
p(y) = y^2 - 2(n+1)y + 2^{\alpha + 1}
\end{align*}
Z Vietových vzorců dostaneme pro kořeny $y_1, y_2$ polynomu $p$:

\begin{align}
y_1y_2 &= 2^{\alpha + 1} \\
y_1 + y_2 &= 2n + 2 \label{eq:Viet}
\end{align}
Tedy $y_1 = 2^a, y_2 = 2^b$ pro nějaké $a,b \geq 0$, bez újmy na obecnosti $a \leq b$. Nyní rozebereme několik případů pro různé hodnoty $a$.

\begin{itemize}
\item[$a = 1$]
Tedy $y_1 = 2$ a $y_2 = 2n$. Po dosazení do polynomu $p$ dostaneme hodnoty $n = 1$ nebo $n = 2$, což jsou nesmyslné hodnoty pro kódy.
\item[$a = 2$]
Tedy $y_1 = 4$ a $y_2 = 2n -2$. Stejným způsobem jako v předchozím bodu dosadíme do $p$ a spočteme $n = 2$ nebo $n = 5$. Pro $n = 5$ dostaneme triviální opakovací kód.
\item[$a \geq 3$]
Z rovnice~\ref{eq:Viet} a faktu, že $a,b\geq 3$ dostaneme pro nějaké $k$:
\begin{align}
2n + 1 &= 2^a + 2^b - 1 \\
&= 8k - 1 \\
(2n + 1)^2 &= 64k^2 - 16k + 1 \label{eq:Mod1}
\end{align}

A dosadíme do rovnice~\ref{eq:SpherePacking2}:
\begin{align}
(2n + 1)^2 = 2^{\alpha + 3} - 7 \label{eq:Mod7}
\end{align}

Pravá strana rovnice~\ref{eq:Mod1} modulo 16 se rovná $1$, zatímco pravá strana rovnice~\ref{eq:Mod7} modulo 16 se rovná $-7$, což je spor, neboť by se pravé strany obou rovnic měly rovnat. Pro $a \geq 3$ tedy neexistuje žádný perfektní kód. \qed

\end{itemize}

\vt Pro $t \leq 3$ a $q > 2$ neexistuje $t$-perfektní kód nad abecedou s $q$ znaky.

Důkaz této části je nejnáročnější, proto ho rozdělíme do několika lemmátek. Hlavní roli budou mít kořeny Lloydova polynomu, které si označíme $\sigma_1, \dots, \sigma_t$ a pro které platí:

\begin{align*}
2 < \sigma_1 < \dots < \sigma_t < n
\end{align*}

Nerovnosti mezi kořeny máme z Lloydovy věty. Po dosazení čísel $0, 1$ a $2$ do Lloydova polynomu, zjistíme, že ani jedno z těchto čísel není kořenem, tedy $2 < \sigma_1$.

\lm $\prod\limits_{i = 1}^{t} \sigma_i = t!q^{\alpha - t}$

\dk Nejprve si vyjádříme součin kořenů pomocí koeficientů Lloydova polynomu.
\begin{align*}
L_t(x) &= a_t x^t + \dots + a_1x + a_0 \\
&= a_t \Bigl(x^t + \frac{a_{t-1}}{a_t}x^{t-1} + \dots + \frac{a_0}{a_t}\Bigr) \\
&= a_t (x - \sigma_1)\dots(x - \sigma_t) \\
&= a_t\Bigl(x^t - \Bigl(\sum \sigma_i\Bigr)x^{t-1} + \dots + (-1)^t \prod \sigma_i\Bigr)
\end{align*}

Tedy máme:
\begin{align}
\prod\limits_{i = 1}^{t} \sigma_i = (-1)^t \frac{a_0}{a_t} \label{eq:ProdRoot}
\end{align}

Nyní spočítáme koeficienty $a_0$ a $a_t$:
\begin{align*}
a_0 &= L_t(0) = \sum\limits_{i = 0}^{t} {n \choose i}(q - 1)^i = q^\alpha \\
a_t &= \sum\limits_{i = 0}^{t} (-1)^i (q - 1)^{t - i} \frac{1}{i!}(-1)^{t-i} \frac{1}{(t-i)!} \\
&= \frac{(-1)^t}{t!}\sum\limits_{i = 0}^{t}(q-1)^{t-i} \frac{t!}{i!(t-i)!} \\
&= \frac{(-1)^t}{t!}\bigl((q - 1) + 1\bigr)^t
\end{align*}

Pro poslední rovnost jsme použili binomickou větu. Po dosazení do rovnice~\ref{eq:ProdRoot} dostaneme:
\begin{align*}
\prod\limits_{i = 1}^{t} \sigma_i = (-1)^t \frac{q^\alpha}{\frac{(-1)^t}{t!}q^t} = t!q^{\alpha-t}
\end{align*}
\qed

\lm $2 \sigma_1 \leq \sigma_t$

\dk Definujme si funkci $f(x) = k \Leftrightarrow x = p^hk$, kde pro $p$ platí $q = p^r$ a $p$ je nesoudělné s $k$. Aplikujme $f$ na součin kořenů:
\begin{align*}
f(\sigma_1)f(\sigma_2) \dots f(\sigma_t) &= f(\sigma_1\sigma_2 \dots \sigma_t) =\\
&= f(t!q^{a-t}) = f(t!) \leq t!
\end{align*}

Uvažme nyní 2 možnosti:
\begin{enumerate}
\item Nechť existují $i \neq j$ takové, že $f(\sigma_i) = f(\sigma_j) = k$, pak:
\begin{align*}
\sigma_i &= p^{h_i}k \\
\sigma_j &= p^{h_j}k \\
\end{align*}
Bez újmy na obecnosti platí $\sigma_i < \sigma_j$ a pak tedy $h_i < h_j$ a $p\sigma_i \leq \sigma_j$. Když všechny nerovnosti dáme dohromady:
\begin{align*}
\sigma_t \geq \sigma_j \geq p\sigma_i \geq 2\sigma_i \geq 2\sigma_1
\end{align*}
\item Nechť jsou tedy všechny $f(\sigma_i)$ různé. Jelikož je jejich součin menší než $t!$, pak se mezi $f(\sigma_1),\dots ,f(\sigma_t)$ vyskytují všechna čísla $1,\dots,t$. Jelikož $t \geq 3$ a $p$ je nesoudělné se všemi čísly $1,\dots,t$, tak $p > 3$. Evidentně musí existovat $i,j$ taková, že:
\begin{align*}
\sigma_i &= p^{h_i} \\
\sigma_j &= 2p^{h_j}
\end{align*}
Rozebereme 2 možnosti:
\begin{enumerate}
\item Nechť $h_j \geq h_i$, pak: $\sigma_t \geq \sigma_j = 2p^{h_j} \geq 2p^{h_i} = 2\sigma_i \geq 2\sigma_1$
\item Nechť $h_i > h_j$, pak: $\sigma_t \geq \sigma_i = p^{h_i} \geq p^{h_j + 1} = \frac{p}{2}\sigma_j \geq 2\sigma_j \geq 2\sigma_1$
\end{enumerate}
\end{enumerate}
\qed

\lm $\sigma_1\sigma_t \leq \frac{8}{9}(\frac{\sigma_1 + \sigma_t}{2})^2$

\dk Celou nerovnost vynásobíme $\sigma^2_1$:
\begin{align*}
\frac{\sigma_t}{\sigma_1} \leq \frac{8}{9}\Bigl(\frac{1 + \frac{\sigma_1}{\sigma_t}}{2}\Bigr)^2
\end{align*}
Provedeme substituci $x = \frac{\sigma_t}{\sigma_1}$:
\begin{align*}
x \leq \frac{8}{9}\Bigl(\frac{1 + x}{2}\Bigr)^2
\end{align*}
Po elementárních úpravách dostaneme:
\begin{align*}
0 \leq \Bigl(x - \frac{1}{2}\Bigr)\Bigl(x - 2\Bigr)
\end{align*}
Což platí, protože z předchozího lemmatu víme, že $x \geq 2$. \qed


\lm $\prod\limits_{i = 1}^t \sigma_i \geq \frac{n^t (q-1)^t}{q^t}\bigl(1 - \frac{t(t-1)}{2n}\bigr)$

\dk Víme, že:
\begin{align*}
\prod\limits_{i = 1}^t \sigma_i = t!q^{\alpha-t}
\end{align*}
Pravou stranu následně budeme upravovat:
\begin{align*}
\frac{t!}{q^t}q^\alpha &= \sum\limits_{i = 0}^{t}(q - 1)^i{n \choose i} \\
&\geq \frac{t!}{q^t} (q - 1)^t \frac{n(n-1)\dots(n-t+1)}{t!}  \\
&= \frac{(q-1)^t}{q^t} n^t \Bigl(1 - \frac{1}{n}\Bigr)\Bigl(1 - \frac{2}{n}\Bigr)\dots\Bigl(1 - \frac{t-1}{n}\Bigr)
\end{align*}
Nyní použijeme vzoreček, který platí pro $x_1,\dots,x_k \in (0,1)$:
\begin{align*}
\prod\limits_{i=1}^{k} (1 - x_i) \geq 1 - \sum\limits_{i=1}^{k} x_i
\end{align*}
Po aplikaci na $(1 - \frac{1}{n})(1 - \frac{2}{n})\dots(1 - \frac{t-1}{n})$, dostaneme:
\begin{align*}
\frac{t!}{q^t}q^\alpha &\geq \frac{n^t (q-1)^t}{q^t}\Bigl(1 - \sum\limits_{i=1}^{t-1} \frac{i}{n}\Bigr) \\
&\geq \frac{n^t (q-1)^t}{q^t}\Bigl(1 - \frac{t(t-1)}{2n}\Bigr) 
\end{align*}
\qed

\lm $\prod\limits_{i = 1}^t \sigma_i \leq \frac{8}{9}\frac{n^t(q-1)^t}{q^t}$

\dk Pro důkaz použijeme již dokázanou nerovnost $\sigma_1\sigma_t \leq \frac{8}{9}(\frac{\sigma_1+\sigma_t}{2})^2$ a nerovnost mezi aritmetickým a geometrickým průměrem:
\begin{align*}
\Bigl(\prod\limits_{i=1}^k x_i \Bigr)^{\frac{1}{k}} \leq \frac{\sum\limits_{i=1}^k x_i}{k}
\end{align*}
Odhadněme tedy součin kořenů:
\begin{align*}
(\sigma_1\sigma_t)(\sigma_2\sigma_3\dots\sigma_{t-1}) &\leq \frac{8}{9}\Bigl(\frac{\sigma_1+\sigma_t}{2}\Bigr)^2\Bigl(\frac{\sigma_2 + \dots + \sigma_{t-2}}{t-2}\Bigr)^{t-2} \\
&\leq \frac{8}{9} \Bigl(\frac{2\frac{\sigma_1 + \sigma_t}{2} + (t-2)\frac{\sigma_2 + \dots + \sigma_{t-2}}{t-2}}{t}\Bigr)^t \\
&=\frac{8}{9}\Bigl(\frac{\sigma_1 + \dots + \sigma_t}{t}\Bigr)^t
\end{align*}




