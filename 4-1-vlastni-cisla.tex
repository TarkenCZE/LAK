\subsection{Vlastní čísla grafu}


\df Nechť $A$ je čtvercová matice. Potom pokud pro nějaké $\lambda$ a $x$ netriviální platí, že $Ax=\lambda x$ říkáme, že $\lambda$ je vlastní číslo $A$ a $x$ je vlastní vektor příslušící k $\lambda$.

\df Spektrum matice $A$ je množina jejích vlastních čísel. Značíme $\Sp(A) = \{ \lambda_1, \ldots, \lambda_n \}$.

\df Podprostorem generovaným vlastním číslem číslem $\lambda$ rozumíme $V_\lambda = \{u | Au = \lambda u \}$. Geometrická násobnost $\lambda$ je poté dimenze tohoto prostoru $V_\lambda$.

\tv $V_\lambda$ je vektorový prostor.

\dk Stačí dokázat uzavřenost. Pro $u,v\in V_\lambda$ počítejme:
\begin{align}
	A(u+v) = Au + Av = \lambda u + \lambda v = \lambda(u+v)
\end{align}
Tedy i $u+v\in V_\lambda$. \qed

\tv Vlastní čísla matice $A$ lze vypočítat jako kořeny rovnice $\det(A - \lambda \cdot E) = 0$.
\dk Z definice počítejme:
\begin{align}
	Au &= \lambda u \\
	Au - \lambda u &= \vec0 \\
	(A-\lambda) u &= \vec0 \\
	\det(A - \lambda E) &= 0 
\end{align}
Přičemž v posledním kroku využíváme faktu, že pro součin netriviálního vektoru s maticí musí být matice singulární, aby mohl vyjít nulový vektor a tudíž můžeme přejít k determinantu. \qed

\df Polynomu $P_A(\lambda) = \det(A - \lambda \cdot E)$ říkáme charakteristický polynom.

\df Násobnosti kořene $\lambda$ v polynomu $P_A$ říkáme {\it algebraická násobnost}.

\vt Nechť $GN(\lambda)$ a $AN(\lambda)$ značí geometrickou, resp. algebraickou násobnost $\lambda$. Potom platí:
\begin{align}
&GN(\lambda) \geq 1 \Leftrightarrow \lambda \in \Sp(A) \Leftrightarrow AN(\lambda) \geq 1\\
\text{a}\qquad &GN(\lambda) \leq AN(\lambda)
\end{align}
\dk {\it (bez důkazu)}

\df Hermitovská transpozice matice $A$ je matice $A^*$, taková, že $A_{ij}^* = \overline{A_{ji}}$.

\df Matice $A \in \C^{n\times n}$ je {\it normální}, pokud $AA^* = A^*A$.

\vt Matice $A$ má ortonormální bázi složenou z vlastních vektorů právě tehdy, když je $A$ normální.

\dk \begin{description}
	\item \uv{$\Rightarrow$} Nechť $x_i$ jsou vlastní vektory příslušející vlastním číslům $\lambda_i$ tvořící ortonormální bázi. Z ortonormality plyne, že $XX^*=E$, kde $X$ má ve sloupcích $x_i$. Podívejme se nyní jak vypadá matice $X^* A X$:
	\begin{align}
		X^* A X = 
		\underbrace{
		\left(\begin{array}{ccc} &\vdots &\\ \hline \hspace{1cm}& x_j^* & \hspace{1cm}\\ \hline&\vdots& \end{array}\right)}_{=X^*} 
		\underbrace{\left(\begin{array}{c|c|c} \raisebox{5mm}{\ } & &\\ \dots& \lambda_i x_i & \dots\\\raisebox{5mm}{\ }  && \end{array}\right)}_{=AX}
		= \left(\begin{array}{ccc}\lambda_1 &&\bigzero \\ &\ddots & \\ \bigzero&&\lambda_n\end{array}\right)
	\end{align}
	Přičemž druhá matice vznikla ze vztahu $Ax=\lambda x$, přičemž jsme vynásobili všechny vektory naráz díky tomu, že byly v matici. Poslední rovnost plyne z pozorování, že na pozici $ij$ nalezneme výraz $x_j^*\lambda_i x_i = x_j^* x_i \lambda_i$ a protože vektory $x_l$ tvoří ortonormální bázi, jsou nula pokud je $i\neq j$ a jedna jinak.

	Nyní víme, že $X^*AX=D$, kde $D$ je nějaká (konkrétní) diagonální matice. Nyní již snadno vypočteme elementárními úpravami:
	\begin{align*}
		X^* A X = D \quad \Rightarrow \quad AX &= XD \quad \Rightarrow A = XDX^* \\
		A\cdot A^* = XD\underbrace{X^*\cdot X}_ED^*X^* = XDD^*X^*
		&= XD^*DX^* = XD^*\underbrace{X^*\cdot X}_EDX^* = A^*\cdot A
	\end{align*}
	Přičemž jediná finta, kterou jsme použili je, že $DD^* = D^*D$, což je zřejmě pravda, protože jsou to diagonální matice.

	\item \uv{$\Leftarrow$} \todo{gavento} byl jen pochybny naznak
\end{description}

\vt Nechť $A_i \in \C^{n\times n}$ a $\forall i,j$ jsou $A_i$ a $A_j$ normální a $A_iA_j = A_j A_i$. Potom existuje společná ortonormální báze z vlastních vektorů.

\dk \todo{Gavento}

\vt Nechť $A$ je hermitovská matice, tedy $A = A^*$. Potom všechna její vlastní čísla jsou reálná.

\dk Víme, že existuje nějaké $D$ diagonální s vlastními čísly na diagonále a $X$, že $X^*AX = D$. Dále počítáme:
\begin{align}
	D^* = (X^*(AX))^* = (AX)^*X = X^*A^*X = X^*AX = D
\end{align}
A komplexní sdružení tedy nesmí udělat žádnou operaci, tedy jsou vlastní čísla reálná. \qed


