\subsection{Reprezentace vzdálenostně regulárních grafů polynomy}


\vt $\dim \A(G) = d+1$, kde $d$ je průměr $G$.\footnote{Průměr grafu je maximální nejkratší vzdálenost přes všechny dvojice vrcholů.}

\dk $A^m = \sum_{i=0}^d Z_{mi}A_i$ \\
$i > m \Rightarrow Z_{mi} = 0$ \\
$A^0 = Z_{0,0} \cdot A_0 = A_0$\\
$A^1 = Z_{1,0} \cdot A_0 + Z_{1,1} \cdot A_1 = A_1$\\
$A^2 = Z_{2,0} \cdot A_0 + Z_{2,1} \cdot A_1 + Z_{2,2} \cdot A_2$\\
$\vdots$\\
$A^d = Z_{d,0} \cdot A_0 + Z_{d,1} + \dots + Z_{d,d}\cdot A_d$

Generujeme celý vektorový prostor polynomů $A$ $\deg \le d$, tedy $\dim \A(G)
\le d+1$. Zároveň ale $A_0, A_1, \dots, A_d$ jsou lineárně nezávislé a proto
$\dim \A(G) = d+1$. 
\qed

\poz $\widetilde \A = \{A_0, A_1, \dots, A_d\}$ tvoří bázi $\A(G)$.

\df Matice $B_h$ pro graf je velikosti $d\times d$, uchovávající parametry $s_{hij}$:
\begin{align}
	(B_h)_{ij} := s_{hij}
\end{align}
Maticí $B$ navíc rozumíme matici $B_1$.

\lm Existuje funkce $f: \A \to \A$, že $f(A_h) = B_h$ a tuto operaci značíme $\widehat A = B$.

\dk Z předchozího lemmatu již máme bázi $\widetilde{\A}$ prostoru $\A$. Ukážeme si tedy, že můžeme přejít k bázi z menších matic $B$. Nejdříve si všiměme, co se děje v následujícím součinu matic:

\begin{align}
	(A_hA_i)_{uv} = \sum_w(A_h)_{uw} \cdot (A_i)_{wv} = s_{hid(u,v)}
\end{align}

Kde zmíněná suma je rozpis maticového násobení pro jednu buňku součinu. Zřejmě přičtu $1$ pokaždé, když pro vrchol $w$ platí, že $d(u,w)=h$ a $d(w,v) = i$, což je přesně definice $s_{hij}$ pro $j = d(u,v)$. Jak takový prvek ještě můžeme vyjádřit (rozepsáním maticového násobení s použitím předchozího vzorce pro buňku)?

\begin{align}
	A_hA_i = \sum_{j=0}^d s_{hij} A_j
\end{align}

Což je vlastně lineární kombinace prvků z báze s koeficienty $s_{hij}$.
Vytvořme tedy novou bázi, například takovou, která bude obsahovat právě tyto
koeficienty. Do řádku $i$ matice $B'_h$ zapíšeme souřadnice součinu $A_hA_i$
vůdči bázi $\widetilde{A}$, tedy $s_{hij}$. Tím získáme matice $B'_h$, které
jsou bazí (vytvořili jsme je zapsáním souřadnic lineárně nezávislých prvků a
tak jsou lineárně nezávislé), která navíc splňuje žádané vlastnosti a tedy
$B'_h = B_h$. \qed



\lm (O sousedech) $B1 = \left(\begin{matrix}
& & & & & & & & \bigzero & \\
& & & & & & & & & \\
& \bigzero & & & & {\smash{\raisebox{.75\normalbaselineskip}{\diagdots{9em}{.5em}}}} & {\smash{\raisebox{1.2\normalbaselineskip}{\diagdots{6.5em}{.5em}}}} & \\
& & & & {\smash{\raisebox{1.3\normalbaselineskip}{\diagdots{6.5em}{.5em}}}} & & \\
\end{matrix}\right)$ je tridiagonální matice. Všechny sloupcové součty jsou stejné a jsou rovny $k$.

\dk Matice je tridiagonální, protože $s_{1,i,j}$ dává smysl jen pro $i \in \{j-1,j,j+1\}$ (z $\Delta$ nerovnosti). Navíc v $j$-tém sloupci je $s_{1,j-1,j} + s_{1,j,j} + s_{1,j+1,j}$, což zahrnuje všechny sousedy $u$, kterých je $k$.
\qed


\lm $B1 = \left(\begin{matrix}
& & & & & & & & \bigzero & \\
& & & & & & & & & \\
& \bigzero & & & & {\smash{\raisebox{.75\normalbaselineskip}{\diagdots{9em}{.5em}}}} & {\smash{\raisebox{1.2\normalbaselineskip}{\diagdots{6.5em}{.5em}}}} & \\
& & & & {\smash{\raisebox{1.3\normalbaselineskip}{\diagdots{6.5em}{.5em}}}} & & \\
\end{matrix}\right)$ je tridiagonální matice $\Rightarrow$ $\forall$ vlastní čísla jsou různá.


